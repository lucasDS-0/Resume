\documentclass{article}
\usepackage[utf8]{inputenc}
\usepackage[paper=a4paper, left=2cm, right=2cm, bottom=2cm, top=2cm]{geometry}
\usepackage{fontawesome}
\usepackage{hyperref}
\usepackage{multicol}


\usepackage{xcolor}
\definecolor{darkred}{rgb}{0.55, 0.0, 0.0}

\hypersetup{
    colorlinks = true,
    breaklinks = true
    linkcolor  = darkred,
    urlcolor   = darkred,
}

\usepackage{amssymb}
\usepackage{enumitem}
\setlist[itemize]{label=\textcolor{darkred}{\tiny $\blacksquare$}}
\renewcommand{\labelenumi}{\textcolor{darkred}{\arabic{enumi}.}}
\renewcommand{\labelenumii}{\textcolor{darkred}{\arabic{enumi}.}}
\renewcommand{\labelenumiii}{\textcolor{darkred}{\arabic{enumi}.}}

\begin{document}
\pagestyle{empty}

\par{\centering
		{\huge Lucas Di Salvo}
	\bigskip\par}

\section*{\faAt ~~ Personal Data} 
\hrule

\
\newline
\

\begin{tabular}{l | l | l}
      \faLinkedin ~ \href{https://www.linkedin.com/in/lucas-di-salvo-6578b915a}{LinkedIn} \
     & \faGithub ~ \href{https://github.com/lucasDS-0}{GitHub}  \
     & \faEnvelope ~ \href{mailto:ldisalvo@dc.uba.ar}{E-Mail: ldisalvo@dc.uba.ar}
\end{tabular}

\section*{\faBook ~~ Education} 
\hrule

\
\newline
\

\begin{tabular}{l l}
    2017 - ongoing  & \textbf{Licenciatura en Ciencias de la Computación}\\
                    & Facultad de Ciencias Exactas y Naturales, Universidad de Buenos Aires \\ \\
    2011 - 2016 & \textbf{Técnico en Computación} \\
                & E.T. N°3 "María Sanchez de Thompson"
\end{tabular}

\section*{\faCubes ~~ Experience} 
\hrule

\
\newline
\

\begin{tabular}{l | l}
    2023 - ongoing  & \textbf{Teaching Assistant at faculty} \\
                            & Preparation and teaching of lectures for undergraduate courses on: \\
                            & Logic and Computability, Programming Paradigms. \\ 
                            & Planning and evaluation of multiple students. \\
                            \\
    2021 - 2022 (7 months) & \textbf{Game Developer} \\
                            & Making of custom UIs and interactions for different game systems, \\
                            & such as a BattlePass, PopUps, Item selectors and more. \\
                            \\
    2020 - 2021 (11 months) & \textbf{Computer science popularizer at faculty} \\
                            & Preparation of courses for high school students about different CS topics, \\ 
                            & stand expositor in science fairs and career path talks. \\
                            \\
    2019 (6 months) & \textbf{Software Developer at "Seincomp Informática"} \\
                    &  Analysis, development and deployment of .NET customized applications, \\ 
                    & with its maintenance and support. \\
                    & Web applications development for internal usage.\\
                    \\
    2019 - 2021     & \textbf{Programming lessons}  \\
                    & Organization and teaching to high school students. \\
                    & The topics include but are not limited to: \\ 
                    & Python programming, OOP, Structured programming, C\#, \\ 
                    & Data structures and Introduction to algorithms.
\end{tabular}

\section*{\faCogs ~~ Tools and Technologies that I've been exposed to} 
\hrule
\

\
    
\begin{multicols}{4}
    \begin{tabular}{l}
        C/C++ \\
        Python \\
        Git \\
        LaTeX
    \end{tabular}

    \begin{tabular}{l}
        SQL \\
        Prolog \\
        ASM \\
        C\#
    \end{tabular}
    
    \begin{tabular}{l}
        VB.NET \\
        Haskell \\
        Bash \\
        Java
    \end{tabular}
    
    \begin{tabular}{l}
        JavaScript \\
        RDBMs \\
        HTML/CSS \\
        Smalltalk
    \end{tabular}
   
\end{multicols}

\newpage

\section*{\faCoffee ~~ Skills} 
\hrule
\

\

\begin{multicols}{2}
    \begin{itemize}
%        \item Team leadership
        \item Technical documentation reading - writing
        \item Self taught
        \item Teaching
        \item Teamwork
        \item Problem Solving
    \end{itemize}
\end{multicols}

\section*{\faLanguage ~~ Languages}
\hrule

\
\newline
\

\begin{tabular}{l | l}
    Spanish & Native \\
    Written English & Advanced \\
    Spoken English & Upper-Intermediate
\end{tabular}

\section*{\faFileCodeO ~~ Projects}
\hrule
\

\begin{itemize}
    \item A project series for \textit{Algorithms and Data Structures 2} (University course), designed to develop implementations for some of the most common data structures and its functionalities (programmed in C++).
    \begin{itemize}
        \item A \href{https://github.com/lucasDS-0/Doubly_Linked_List}{Doubly Linked List}.
        \item A \href{https://github.com/lucasDS-0/BST_on_a_set}{Binary Search Tree}, implemented on a Set.
        \item A \href{https://github.com/lucasDS-0/Map_on_a_trie}{Map}, implemented on a Trie.
        \item A \href{https://github.com/lucasDS-0/Priority_queue_on_a_heap}{Priority Queue}, implemented on a Heap.
    \end{itemize}
    \item A team projects series for \textit{Algorithms and Data Structures 3} (University course), designed to research, analyze and develop algorithms using different programming techniques to address complex problems (programmed in C++ and Python3 using Jupyter notebook).
    \begin{itemize}
        \item An analysis on the \href{https://github.com/lucasDS-0/Subset_sum}{Subset Sum} problem. The aim of this projects was to ensure the correct procedure to develop solutions by using brute force, backtracking and dynamic programming for the problem at hand, and analyze the effectiveness and efficacy of said techniques, in great detail.
        \item A project about understanding and developing heuristics over the \href{https://github.com/lucasDS-0/MIC_Polytope}{Maximum Impact Coloring Polytope} problem.
    \end{itemize}
        \item A small \href{https://github.com/lucasDS-0/hFetch}{tool} to fetch system information written in Haskell using System.IO to practice monads (both in do-notation and the vanilla monadic notation).
    \item This same \href{https://github.com/lucasDS-0/Resume}{resume}, made in \LaTeX.
\end{itemize}



\section*{\faLaptop ~~ Courses and Certifications} 
\hrule

\
\newline
\

\begin{tabular}{l l l}

    Problem Solving (Basic) & Hackerrank 
                            & (\href{https://www.hackerrank.com/certificates/8d1db5b492de}{See credential}) \\
    Python (Basic)  & Hackerrank 
                    & (\href{https://www.hackerrank.com/certificates/46699d054d5a}{See credential}) \\ 
\end{tabular}

\section*{\faHeartO ~~ Interests} 
\hrule

\
\newline
\

\noindent I wish to learn more about science and technology with their applications in society, while learning from the industry and academia alike, to nurture myself.

\end{document}
